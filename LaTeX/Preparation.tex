\documentclass{article}
\usepackage[a4paper, total={6in, 8in}]{geometry}
\usepackage[utf8]{inputenc}
\usepackage[T2A]{fontenc}
\usepackage[russian]{babel}

\usepackage[table,xcdraw]{xcolor}
\usepackage{graphicx}
\usepackage{placeins}

\begin{document}
\title{Анализ доли высокочастотной торговли и ее влияния на волатильность курса рубля на валютном рынке ММВБ}
\author{}
\date{}
\maketitle
\section{Введение}
	\subsection{Риски высокочастотной торговли - волатильность}
	\subsection{Плюсы высокочастотной торговли - ликвидность, спреды}
\section{Обзор литературы}
\section{Методология}
	\subsection{Определение высокочастотной активности}
	\subsection{Расчет внутридневной волатильности}
	\subsection{Расчет показателя ликвидности, спредов}
\section{Доля высокочастотной торговли}
\section{Результаты анализа причинно-следственной связи между HFT активностью и волатильностью}
\section{Результаты анализа влияния HFT на ликвидность, спреды}
\section{Инструменты Московской Биржи для регулирования высокочастотной активности}
\section{Заключение и выводы}

\newpage

\section*{Показатели}

	\subsection*{Order-to-trade}
		\noindent
		\textbf{Метрика:} Количество заявок (выставление заявки, изменение заявки, удаление заявки), поделенное на количество сделок.

		\noindent
		\textbf{Мотивация:} Высокочастотная торговля характеризуется большим количеством заявок за короткие промежутки времени на разные уровнях цен. Высокое значение коэффициента OTR предполагает автоматизированность, скорость и высокой степень толерантности риска.

	\subsection*{Доля оборота торговли за день}
		\noindent
		\textbf{Метрика:} один минус остаток позиции, поделенное на общий оборот.
		
		\noindent
		\textbf{Мотивация:} Данный показатель отображает, насколько внутридневные позиции ликвидируются перед закрытием торговли. Высокочастотные торговцы имеют тенденцию к закрытию большинства открытых позиций за день, поэтому их позиции O/N достаточно малы. Этот показатель отделяет высокочастотных торговцев от обычных торговых алгоритмов, которые преимущественно торгуют в одну сторону в течение дня.

		\noindent
		\textbf{Проблема:} На нашем рынке большинство участников торговли, занимающихся высокочастотной торговлей, занимаются не только ей. Способа отделить эти обычные операции от высокочастотных по имеющейся базе данных не представляется возможным.

	\subsection*{Полный оборот за день}
		\noindent
		\textbf{Метрика:} Общая сумма покупки плюс общая сумма продажи.

		\noindent
		\textbf{Мотивация:} Для высокочастотной торговли типичны низкомаржинальные стратегии, что означает, что HFT торговцы должны быть достаточно активны на рынке, чтобы получать доход. Высокочастотные торговцы обычно имеют высокий оборот за день.

		\noindent
		\textbf{Проблема:} Как и в предыдущем пункте, на нашем рынке большинство участников торговли, занимающихся высокочастотной торговлей, занимаются не только ей. Следовательно, показатель может быть сильно завышен за счет фундаментальных операций.

	\subsection*{Количество быстрых сообщений}
		\noindent
		\textbf{Метрика:} Абсолютное количество заявок в течение 40-миллисекундного окна, отправленных после определенного события (изменение или отмена существующей заявки в журнале заявок в течение 40 миллисекунд после предыдущего подобного действия; или заявка с лучшей ценой, отправленная после некоторой паузы на рынке).

		\noindent
		\textbf{Мотивация:} Высокочастотная торговля характеризуется большим количества сообщений в течение 40-миллисекундного окна.  Не существует какого-то определенного метода, используемого высокочастотными торговцами для воздействия на журнал заявок - некоторые снимают и выставляю новые заявки, другие последовательно изменяют свои имеющиеся заявки.

		\noindent
		\textbf{Необходимо:} База данных заявок со временем по миллисекундам. 

	\subsection*{Период держания}
		\noindent
		\textbf{Метрика:} Взвешенное по объему время держания позиции.

		\noindent
		\textbf{Мотивация:} Высокочастотная торговля характеризуется множественным открытием и закрытием позициям в течение дня. Частые, маленькие и изменяющиеся позиции являются ключевым элементом данного типа торговли. Высокочастотные торговцы типично имеются короткие период держания.
		
		\noindent
		\textbf{Необходимо:} База данных заявок со временем по миллисекундам. Можно попробовать по имеющейся базе данных, но многие, есть вероятность, что многие сделки совершены внутри одной секунды.

	\subsection*{Коэффициент at-best}
		\noindent
		\textbf{Метрика:} Количество заявок, размещенных по лучшей цене плюс количество заявок, оцененных ``по рынку'', поделенное на общее количество размещенных заявок.

		\noindent
		\textbf{Мотивация:} Данный показатель отображает долю заявок, размещенных по лучшей цене. Высокочастотные торговцы активно участвуют в оценке активов, поэтому многие заявки должны быть размещены по рыночной цене, или близкой к ней. Активные высокочастотные торговцы типично имеют высокое значение коэффициента at-best.

		\noindent
		\textbf{Необходимо:} База данных журнала заявок.

\section*{Влияния высокочастотной торговли на рынок}
	
	\subsection*{Вклад в глубину журнала заявок}

		\begin{itemize}
			\item Средняя доля объема по лучшим ценам bid и ask, предоставленная высокочастотными торговцами;
			\item Средняя доля объема по лучшим ценам bid и ask, плюс 2 шага цены, предоставленная высокочастотными торговцами.
		\end{itemize}

	\subsection*{Непродолжительные заявки}

		Заявки высокочастотных трейдеров, выставляемых на рынок на непродолжительное время хотя и предоставляют ликвидность, однако для других инвесторов не представляется возможным ей воспользоваться.

\end{document}
