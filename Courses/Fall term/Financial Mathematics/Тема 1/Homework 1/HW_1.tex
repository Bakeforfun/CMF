% -*- program: xelatex -*-
\documentclass[a4paper,12pt]{article}
%---------------
% Обычные параметры для XeLaTeX
%---------------
\usepackage{amsfonts,longtable,amsmath,amssymb,mathtools}
\usepackage[cm-default]{fontspec}
\usepackage{polyglossia} % Языки: русский, английский
%\usepackage{fourier}
%\DeclareMathAlphabet{\pazocal}{OMS}{zplm}{m}{n}
%\newcommand{\La}{\mathcal{L}}
%\newcommand{\Lb}{\pazocal{L}}
\usepackage{xunicode}
\usepackage{xltxtra}
\defaultfontfeatures{Ligatures=TeX}
\usepackage{mathrsfs}
%\usepackage{unicode-math}
%\unimathsetup{math-style=TeX}
\newfontfamily\cyrillicfont[Mapping=tex-text]{Times New Roman}
\newfontfamily{\cyrillicfonttt}{Times New Roman}
\setmainfont[Mapping=tex-text]{Times New Roman}
\setdefaultlanguage[spelling=modern]{russian}

\usepackage{xecyr}
\defaultfontfeatures{Scale=MatchLowercase}

%---------------
% Параметры для PdfLaTeX?
%---------------
%\usepackage{cmap} % Улучшенный поиск русских слов в полученном pdf-файле
%\usepackage[T2A]{fontenc} % Поддержка русских букв
%\usepackage[utf8]{inputenc} % Кодировка utf8
%\usepackage{fontenc}
%\usepackage[english, russian]{babel} % Языки: русский, английский
%\usepackage{pscyr}
%\setmathfont{XITS Math}
%\setmainfont[Mapping=tex-text]{Calibri}
%\setotherlanguage{english}
%---------------
% Параметры для документа
%---------------
\usepackage[left=2cm,right=2cm,top=2cm,bottom=2cm,bindingoffset=0cm]{geometry}
\usepackage[some]{background}
% Не используемые из-за пакета geometry параметры
%\oddsidemargin=0pt
%\textwidth=460pt
%\topmargin=-30pt
%\voffset=0pt
%\headheight=0pt
%\headsep=30pt
\textheight=700pt
\usepackage{setspace} % Междустрочные интервалы
\onehalfspacing % Установка полуторного интервала
\usepackage{hyperref} % Гиперссылки с отключенным параметром рамок вокруг ссылок
\usepackage{xcolor} % Поддержка различных цветов
\hypersetup{
	colorlinks,
	linkcolor={red!50!black},
	citecolor={blue!50!black},
	urlcolor={blue!80!black}
}
\usepackage{wrapfig} % Плавающие изображения
\usepackage{multicol} % Поддержка объединения колонок в таблицах
\usepackage{multirow} % Поддержка объединения рядов в таблицах 
\columnsep=30pt % Что это?
\makeatletter % Что это?
\usepackage[strict]{changepage}
\usepackage{booktabs}
\newcommand{\minitab}[2][l]{\begin{tabular}{#1}#2\end{tabular}}
%\usepackage{slashbox} % Ячейка с косой чертой в таблицах
\usepackage{listings} % Окружение для исходного кода
\usepackage{epigraph} % Поддержка эпиграфов
%\usepackage{indentfirst} % Поддержка красной строки
\usepackage{yfonts} % Поддержка старых готических шрифтов
\usepackage{lettrine} % Пакет для каллиграфических первых букв в абзацах (буквицы?)
%\usepackage{scrextend} % Какой-то сложный пакет
\usepackage{enumitem} % Пакет для управления окружениями списков
\setlist{nosep} % or \setlist{noitemsep} to leave space around whole list
\usepackage{framed} % Прямоугольники, линии
\usepackage{rotating} % Пакет для поворота объектов (например, таблиц)
%---------------
% Изменение команд
%---------------
\renewcommand \thesection {\arabic{section}}
\renewcommand \thesubsection {\arabic{section}.\arabic{subsection}}
\renewcommand{\@listi}{
	% вертикальные промежутки:
	\topsep=0pt % вокруг списка
	\parsep=0pt % между абзацами
	\itemsep=0pt % между пунктами
	% горизонтальные промежутки:
	\itemindent=0pt % абзацный выступ
	\labelsep=1ex % расстояние до метки
	\leftmargin=\parindent % отступ слева
	\rightmargin=0pt} % отступ справа
%---------------
% Изменение колонтитулов
\usepackage{fancyhdr} % Пакет для изменения колонтитулов
\pagestyle{fancy}
\fancypagestyle{liststyle}
{
	\fancyhf{}
	\renewcommand\headrulewidth{0pt}
%\rhead{А.С. Пушкин}
%\lhead{Финансовая математика, домашнее задание №0}
}

\makeatletter
\renewcommand\tableofcontents{%
	\if@twocolumn
	\@restonecoltrue\onecolumn
	\else
	\@restonecolfalse
	\fi
	\pagestyle{liststyle}
	\@starttoc{toc}%
	\if@restonecol\twocolumn\fi
}
\makeatother

%\rhead{А.С. Пушкин}
\chead{}
%\lhead{Финансовая математика, домашнее задание №0}
\lfoot{}
\cfoot{\normalsize\thepage}
\rfoot{}
\renewcommand{\headrulewidth}{0pt}
\renewcommand{\footrulewidth}{0pt}
\def\changemargin#1#2{\list{}{\rightmargin#2\leftmargin#1}\item[]}
\let\endchangemargin=\endlist
%---------------
%\renewcommand{\alph}[1]{\asbuk{#1}} % костыль для кирилической нумерации
% вместо латинской
%\setenumerate[1]{label=\alph*),nolistsep}
\setenumerate[1]{label=\bf\arabic*.}
\setenumerate[2]{label=\arabic*, fullwidth, itemindent=\doubleparindent,listparindent=\parindent}
%\setenumerate[1]{nolistsep, itemsep=0.3cm,parsep=0pt}
\setitemize[1]{nolistsep}
%---------------
\newenvironment{quotationb}%
{\begin{leftbar}\begin{quotation}}%
		{\end{quotation}\end{leftbar}} % for adjustwidth environment
% Важное окружение formal
%---------------
\definecolor{formalshade}{rgb}{0.95,0.95,1}
\newenvironment{formal}{%
	\def\FrameCommand{%
		\hspace{1pt}%
		{\color{black}\vrule width 1.5pt}%
		{\color{white}\vrule width 4pt}%
		\colorbox{white}%
	}%
	\MakeFramed{\advance\hsize-\width\FrameRestore}%
	\noindent\hspace{-4.55pt}% disable indenting first paragraph
	\begin{adjustwidth}{}{7pt}%
		\vspace{2pt}\vspace{2pt}%
	}
	{%
		\vspace{2pt}\end{adjustwidth}\endMakeFramed%
}
% Cписок с точками без интервалов между пунктами
%---------------
\newenvironment{compactlist}{
	\begin{list}{{$\bullet$}}{
			\setlength\partopsep{0pt}
			\setlength\parskip{0pt}
			\setlength\parsep{0pt}
			\setlength\topsep{0pt}
			\setlength\itemsep{0pt}
		}
	}{
\end{list}
}
%---------------
% Пакеты для построения графиков и параметры для них
%---------------
\usepackage{pgf,tikz} % Чертежи и графики
\newcommand\LM{\ensuremath{\mathit{LM}}} % Реликты старых графиков
\newcommand\IS{\ensuremath{\mathit{IS}}} % Реликты старых графиков
\usetikzlibrary{arrows,calc,intersections,patterns,decorations.fractals}
\usepackage{relsize} % Относительный размер в графиках

% Важная команда для получения координат
%---------------
\newcommand{\gettikzxy}[3]{%
	\tikz@scan@one@point\pgfutil@firstofone#1\relax
	\edef#2{\the\pgf@x}%
	\edef#3{\the\pgf@y}%
}
\title{Курс <<Введение в основы финансовой математики>>.\\Домашнее задание №1.}
\date{}
\author{В.П. Закатов | Mandelbrot}

\begin{document}
\maketitle

\subsection*{Задача 1}
\begin{formal}
Вычислить мартингальные вероятности для одношаговой биномиальной модели.
\end{formal}
\textbf{\large Решение:}

В одношаговой биномиальной модели мы имеем два фиксированных момента времени $t=0$ и $t=1$, а также два актива: облигация (банковский счет с безрисковой ставкой $R$ по депозиту) и акция (рискованный актив). 

Цена облигации определена как
\begin{equation*}
	B_0 = 1
\end{equation*}
\begin{equation*}
	B_1 = 1 + R
\end{equation*}

Цена акции определена как 
\begin{equation*}
	S_0 = s
\end{equation*}
\begin{equation*}
	S_1 = sZ,
\end{equation*}
где
\begin{equation*}
	Z = \begin{cases} u \text{, с вероятностью $p_u$} \\ d \text{, с вероятностью $p_d$} \end{cases}
\end{equation*}

Для того, чтобы на рынке отсутствовали арбитражные возможности, должно выполняться следующее неравенство:

\begin{equation}
	d < (1+R) < u
\end{equation}

Легко показать, что при невыполнении данного условия можно построить портфель, который будет гарантированно приносить прибыль в момент времени $t = 1$ при нулевой стоимости в момент времени $t = 0$. Например, если $d \geq (1+R)$, тогда, имея нулевой капитал в периоде $t=0$, можно занять деньги на денежном рынке (продав облигацию) и использовать полученные средства на покупку акции. Тогда, даже при неблагоприятном исходе у инвестора будет достаточно средств в момент $t=1$, чтобы расплатиться с долгом по облигации, а также будет ненулевая вероятность получения прибыли, так как $u > d \geq (1+R)$. Данная ситуация является арбитражной и должна быстро исчезнуть по закону спроса и предложения, когда достаточное количество инвесторов ей воспользуются. Ситуация $u \leq (1+R)$ аналогична предыдущей, инвестору необходимо продать акцию и использовать средства на покупку облигации.

Таким образом, предполагая, что неравенство (1) выполняется, возможно найти такую пару чисел ($q_u, q_d$), так что ($1+R$) можно записать как следующую выпуклую комбинацию:
\begin{equation}
	(1+R) = q_u u + q_d d,
\end{equation}
где $q_u, q_d \geq 0$ и $q_u + q_d = 1$. Получаем
\begin{equation*}
	(1+R) = q_u u + (1-q_u) d
\end{equation*}
\begin{equation}
	q_u = \frac{(1+R) - d}{u-d}
\end{equation}
тогда
\begin{equation}
	q_d = \frac{u - (1+R)}{u-d}
\end{equation}

Формулы (3) и (4) являются мартингальными вероятностями.

\subsection*{Задача 2}
\begin{formal}
Рассмотрим некоторый дериватив $D_0$ в одношаговой модели ($B$; $S$) рынка (модели, описанной в лекции) без арбитражных возможностей. Зададим реплицирующую этот дериватив стратегию $\phi$ (напомним, что ее cash flow по определению соответствует таковому у дериватива). Покажите, что если $D_0 \neq V_0(\phi)$, то есть, цена дериватива не соответствует стоимости реплицирующего портфеля в нулевой момент времени, то на рынке возможен арбитраж для расширенной стратегии ($\hat{\phi}$) = ($\phi$;$\gamma$), где $\gamma$ — количество единиц дериватива в расширенном портфеле.
\end{formal}
\textbf{\large Решение:}

Дериватив $D_0$ в одношаговой модели ($B$; $S$) рынка без арбитражных возможностей имеет функцию выплат $\Phi$. Если данное обязательство достижимо в момент времени $t=1$, то существует такой портфель $h = (x,y)$, что
\begin{equation*}
	V_1^h = \Phi(u) \text{, если $Z=u$}
\end{equation*}
\begin{equation*}
	V_1^h = \Phi(d) \text{, если $Z=d$}
\end{equation*}

Данные уравнения можно записать следующей системной уравнений:
\begin{equation*}
	\begin{cases}
		(1+R)x+suy = \Phi(u)\\
		(1+R)x+sdy = \Phi(d)
	\end{cases}
\end{equation*}

Данная система имеет единственное решение (предполагая $d<u$) и можно записать
\begin{equation*}
	x = \frac{1}{1+R} \frac{u\Phi(d)-d\Phi(u)}{u-d}
\end{equation*}
\begin{equation*}
	y = \frac{1}{s} \frac{\Phi(u)-\Phi(d)}{u-d}
\end{equation*}

Тогда наш реплицирующий портфель $h=(x,y)$ должен иметь следующую стоимость в момент времени $t=0$
\begin{equation*}
	V_0^h = x+sy = \frac{1}{1+R} \bigg( \frac{(1+R) - d}{u-d} \Phi(u) + \frac{u - (1+R)}{u-d} \Phi(d) \bigg) = \frac{1}{1+R} E^Q [D_1] = D_0
\end{equation*}

Мы показали, что если существует портфель $h = (x,y)$, полностью реплицирующий выплаты по деривативу в момент $t=1$, то стоимость дериватива $D_0$ и портфеля $V_0^h$ должны быть равны в момент времени $t=0$.

Если $D_0 \neq V_0(\phi)$, тогда можно будет составить такой портфель ($\hat{\phi}$) = ($\phi$;$\gamma$) (где $\gamma$ — количество единиц дериватива в расширенном портфеле), состоящий из дериватива и реплицирующего порфтеля, который принесет арбитражную прибыль.

К примеру, предположим $D_0 > V_0(\phi)$. Тогда возможно продать $\frac{V_0(\phi)}{D_0}$ единиц дериватива и купить 1 единицу реплицирующего портфеля. Тогда, в момент $t=0$ стоимость такого портфеля будет равна нулю:
\begin{equation}
	V_0(\phi) - \frac{V_0(\phi)}{D_0} D_0 = 0
\end{equation}

В момент времени $t=1$ мы имеем 
\begin{equation}
	V_1(\phi) - \frac{V_0(\phi)}{D_0} D_1 = \Big( 1 - \frac{V_0(\phi)}{D_0} \Big)
\end{equation}

Так как $D_0 > V_0(\phi)$, то $\frac{V_0(\phi)}{D_0} < 1$. Следовательно, мы нулевой стоимости порфтеля в начальный момент времени, мы имеет положительную стоимость порфтеля в момент $t=1$ с вероятностью 1. Данная ситуция является арбитражной. 

Случай $D_0 < V_0(\phi)$ рассматривается аналогично, где мы купим 1 единицу дериватива и продадим $\frac{D_0}{V_0(\phi)}$.


\subsection*{Задача 3}
\begin{formal}
Докажите, что если в одношаговой биномиальной модели с вероятностью 1 выполняется $V_0(X) \neq X$, иначе говоря, стоимость портфеля в нулевой момент времени не равна суммарной стоимости его компонентов ($B$, облигаций и $S$, акций), то существует безрисковый доход.

\end{formal}
\textbf{\large Решение:}

Для подхода к данной задаче мы можем рассматривать стоимость порфеля в момент времени $t=0$ как некоторый дериватив, который реплицирует выплаты по составным частям портфеля в момент времени $t=1$. Данную задачу можно свести к предыдущей, показав, что если $V_0(X) \neq X$, тогда всегда можно составить порфель, состоящий из изначального портфеля и его составлных частей, где будут куплены относительно недооцененные части и проданы относительно переоцененные части. Такой портфель всегда будет арбитражным, пока цены не вернутся к равновесным значением и не будет выполнятся условие $V_0(X) = X$.


\subsection*{Задача 4}
\begin{formal}
Цена акции в настоящий момент составляет \$40. Известно, что в конце месяца она составит либо \$42, либо \$38. Ставка дисконтирования равна 8\% в год, считать, что она разбивается на 12 месячных ставок. Найти стоимость 1-месячного европейского опциона типа ``колл'' с ценой страйка, составляющей \$39 с помощью:
\begin{itemize}
	\item Создания реплицирующего портфеля.
	\item Расчета риск-нейтральных вероятностей в одношаговой биномиальной модели.
\end{itemize}
\end{formal}
\textbf{\large Решение:}

Мы имеем $s = 40$, $u = \frac{42}{40}$, $d = \frac{38}{40}$, $R = 0.08 / 12$, $K = 38$. Выплата опциона колл опрелена как $D_1 = \max(S_1 - K, 0)$ и равна в момент времени $t=1$ $\Phi(u) = 3$ и $\Phi(d) = 0$ в благоприятном и неблагоприятном исходах, соответсвенно.

1. Найдем стоимость опциона колл с помощью реплицируешего портфеля. Рассмотрим портфель $h = (x,y)$, где 
\begin{equation*}
	x = \frac{1}{1+R} \frac{u\Phi(d)-d\Phi(u)}{u-d} = -28.31
\end{equation*}
\begin{equation*}
	y = \frac{1}{s} \frac{\Phi(u)-\Phi(d)}{u-d} = 0.75
\end{equation*}

Стоимость портфеля в начальный момент времени составляет
\begin{equation*}
	V_0^h = x+sy = -28.31 + 40*0.75 = 1.68
\end{equation*}

2. Найдем стоимость опциона колл с помощью риск-нейтральных вероятностей в одношаговой биномиальной модели. Рассчитаем мартингальные вероятности:

\begin{equation*}
	q_u = \frac{(1+R) - d}{u-d} = 0.5(6)
\end{equation*}
тогда
\begin{equation*}
	q_d = \frac{u - (1+R)}{u-d} = 0.4(3)
\end{equation*}

Соответствующая стоимость опциона колл будет равна:
\begin{equation*}
	D_0 = \frac{1}{1+R} \Big( q_u \Phi(u) + q_d \Phi(d) \Big) = 1.68
\end{equation*}

\subsection*{Задача 5 (бонусная)}
\begin{formal}
Найти реплицирующую стратегию и оценить стоимость европейского платежного обязательства в биномиальной модели. Выплата по такому обязательству основывается только на состоянии рынка на момент даты экспирации, — таким образом, в одношаговой биномиальной модели реализация платежа по обязательству будет происходить в момент времени $t = 1$, а его размер будет зависеть от того, в какую сторону ``сходит'' рынок, задаваясь, таким образом, величинами $H^u$ и $H^d$.
\end{formal}
\textbf{\large Решение:}

Обозначим стоимость европейского платежного обязательства за $H$. Тогда в условиях полного рынка стоимость опциона в момент $t=0$ можно записать
\begin{equation*}
	H = \alpha B_0 + \beta S_0
\end{equation*}

В момент времени $t=1$ стоимость опциона составит
\begin{equation*}
	\begin{cases}
		\alpha(1+R) + \beta S_0 u = H^u\\
		\alpha(1+R) + \beta S_0 d = H^d
	\end{cases}
\end{equation*}

Домножив первое равенство на $d$, второе на $u$, после вычитаняи одного из другого получаем
\begin{equation*}
	\alpha = \frac{1}{1+R} \frac{u H^d-d H^u}{u-d}
\end{equation*}
\begin{equation*}
	\beta = \frac{1}{S_0} \frac{H^u-H^d}{u-d}
\end{equation*}

Отсюда получаем стоимость опциона колл в момент времени $t=0$:
\begin{equation*}
	H_0(\phi) = \alpha B_0 + \beta S_0 = \frac{1}{1+R} \bigg( \frac{(1+R) - d}{u-d} H^u + \frac{u - (1+R)}{u-d} H^d \bigg) = \frac{1}{1+R} E^Q \bigg[\frac{H}{1+R}\bigg]
\end{equation*}


\end{document}